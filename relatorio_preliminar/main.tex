%%
%% This is file `sample-sigchi.tex',
%% generated with the docstrip utility.
%% but modified by the faculty @ DI/FCUL
%% The original source files were:
%%
%% samples.dtx  (with options: `sigchi')
%% 
%% IMPORTANT NOTICE:
%% 
%% For the copyright see the source file.
%% 
%% Any modified versions of this file must be renamed
%% with new filenames distinct from sample-sigchi.tex.
%% 
%% For distribution of the original source see the terms
%% for copying and modification in the file samples.dtx.
%% 
%% This generated file may be distributed as long as the
%% original source files, as listed above, are part of the
%% same distribution. (The sources need not necessarily be
%% in the same archive or directory.)
%%
%% The first command in your LaTeX source must be the \documentclass command.
\documentclass[sigplan]{acmart}
\usepackage{subcaption}
\usepackage{caption}
\usepackage{array}
\usepackage{multirow}
%%
%% \BibTeX command to typeset BibTeX logo in the docs
\AtBeginDocument{%
  \providecommand\BibTeX{{%
    \normalfont B\kern-0.5em{\scshape i\kern-0.25em b}\kern-0.8em\TeX}}}

\setcopyright{none}
\copyrightyear{}
\acmYear{}
\acmDOI{}

\acmConference[]{Introduction to Research }{December 2021}{Lisbon}
\acmBooktitle{}
\acmPrice{}
\acmISBN{}


%%
%% Submission ID.
%% Use this when submitting an article to a sponsored event. You'll
%% receive a unique submission ID from the organizers
%% of the event, and this ID should be used as the parameter to this command.
%%\acmSubmissionID{123-A56-BU3}

%%
%% The majority of ACM publications use numbered citations and
%% references.  The command \citestyle{authoryear} switches to the
%% "author year" style.
%%
%% If you are preparing content for an event
%% sponsored by ACM SIGGRAPH, you must use the "author year" style of
%% citations and references.
%% Uncommenting
%% the next command will enable that style.
%%\citestyle{acmauthoryear}

%%
%% end of the preamble, start of the body of the document source.
\begin{document}

%%
%% The "title" command has an optional parameter,
%% allowing the author to define a "short title" to be used in page headers.
\title{Title of Your Work}

%%
%% The "author" command and its associated commands are used to define
%% the authors and their affiliations.
%% Of note is the shared affiliation of the first two authors, and the
%% "authornote" and "authornotemark" commands
%% used to denote shared contribution to the research.

\author{Your Name - Your Number}
\affiliation{%
 \institution{ Estudo Orientado em $******$ \\ Mestrado em \{nome do Mestrado\} \\ Faculdade de Ciências, Universidade de Lisboa}
 }
\email{youremail@fc.ul.pt}


\renewcommand{\shortauthors}{Your Name - Your Number}

%%
%% The abstract is a short summary of the work to be presented in the
%% article.
\begin{abstract}
  your abstract (about 10 lines).
\end{abstract}


%%
%% Keywords. The author(s) should pick words that accurately describe
%% the work being presented. Separate the keywords with commas.
\keywords{5 keywords}


%%
%% This command processes the author and affiliation and title
%% information and builds the first part of the formatted document.
\maketitle

\section{Introduction}

Here you should motivate your work. 

What is the context? \\

What is the problem? \\

Why is it important? \\

What data and methods are you thinking about using to tackle it? \\

How is your approach different from what was already done (if something was done). \\

How is this document organised? \\

\section{Background} \label{sec:background}

This section should contain any information needed to understand the problem you are tackling.

\subsection{You choose the sections} 



\subsection{Add as many as you need} 


\section{Related Work} \label{sec:relatedwork}

This is the time to do a literature review! \\

What is the state of the art on the topic you are working? \\

Previous work with same data, similar work with other data, ... \\

Problems tackled, data science approaches used, pros and cons, ... \\ 


Example of a reference\cite{lamport1994}. This was a book, but all other relevant works in papers \cite{turing1937} could be added as well.

\section{Data \{if necessary\} } \label{sec:data}

Here you should describe your data in as much detail as possible. \\ 

You can describe raw data and any pre-processing need for your work. \\

You can have a section on exploratory data analysis. \\

\section{Methods} \label{sec:methods}

Here you should describe in as much details as possible the problem and your plan to tackle it. \\

What are the methods you are planning to use, or already started to use, to tackle your problem. \\

This should be based on related word, your understanding of the problem and eventually preliminary exploratory data analysis or preliminary results.

\subsection{You decide on the subsection} 

\section{Preliminary Results (Optional)} \label{sec:preliminaryresults}

Here you can include preliminary results if you already have them.


\section{Forthcoming Work} \label{sec:forthcomingwork}

Here you should write the conclusion of the preliminary work and the goals for the remaining period.


%%
%% The next two lines define the bibliography style to be used, and
%% the bibliography file.
\bibliographystyle{ACM-Reference-Format}
\bibliography{sample-base}




\end{document}

\endinput
%%
%% End of file `sample-sigchi.tex'.
